% standard macros for WEB listings (in addition to PLAIN.TEX)
% pdfTeX adjustments maintained by Han The Thanh <hanthethanh@gmx.net>.
% XeTeX adjustments maintained by Khaled Hosny <khaledhosny@eglug.org>.
% macro integration à la CWEB by Andreas Scherer <andreas_github@freenet.de>.
\ifx\renewenvironment\undefined\else\endinput\fi % LaTeX will use other macros
\xdef\fmtversion{\fmtversion+WEBMAC4.1} % identifies current set of macros
\newif\ifpdf
\ifx\pdf+\pdftrue\fi % plain TeX + xdvipdfmx '\let\pdf+'
% Uncomment the following line if you want PDF goodies to be the default
%\ifx\pdf-\else\pdftrue\fi
\def\pdflinkcolor{0 0 1} % the RGB values for hyperlink color
\newif\ifpdftex
\ifx\pdfoutput\undefined \pdftexfalse \else\ifnum\pdfoutput=0 \pdftexfalse
\else \pdftextrue \pdfoutput=1 \pdfcompresslevel=9 \input pdfcolor
  \def\Black{\pdfliteral{0 g 0 G}} % use rgb colors for direct PDF output too
  \def\Blue{\pdfliteral{0 0 1 rg 0 0 1 RG}}
\fi\fi
\newif\ifacro \ifpdf\acrotrue\fi \ifpdftex\acrotrue\fi

\parskip 0pt % no stretch between paragraphs
\parindent 1em % for paragraphs and for the first line of Pascal text

\font\eightrm=cmr8 \let\sc=\eightrm % NOT a caps-and-small-caps font!
\let\mainfont=\tenrm
\font\titlefont=cmr7 scaled\magstep4 % title on the contents page
\font\ttitlefont=cmtt10 scaled\magstep2 % typewriter type in title
\font\tentex=cmtex10 % TeX extended character set (used in strings)
\fontdimen7\tentex=0pt % no extra space after punctuation

\def\\#1{\hbox{\it#1\/\kern.05em}} % italic type for identifiers
\def\|#1{\hbox{$#1$}} % one-letter identifiers look a bit better this way
\def\&#1{\hbox{\bf#1\/}} % boldface type for reserved words
\def\.#1{\hbox{\tentex % typewriter type for strings
  \let\\=\BS % backslash in a string
  \let\'=\RQ % right quote in a string
  \let\`=\LQ % left quote in a string
  \let\{=\LB % left brace in a string
  \let\}=\RB % right brace in a string
  \let\~=\TL % tilde in a string
  \let\ =\SP % space in a string
  \let\_=\UL % underline in a string
  \let\&=\AM % ampersand in a string
  #1}}
\def\#{\hbox{\tt\char`\#}} % parameter sign
\def\${\hbox{\tt\char`\$}} % dollar sign
\def\%{\hbox{\tt\char`\%}} % percent sign
\def\^{\ifmmode\mathchar"222 \else\char`^ \fi} % pointer or hat
% circumflex accents can be obtained from \^^D instead of \^
\def\AT!{@} % at sign for control text

\chardef\AM=`\& % ampersand character in a string
\chardef\BS=`\\ % backslash in a string
\chardef\LB=`\{ % left brace in a string
\def\LQ{{\tt\char'22}} % left quote in a string
\chardef\RB=`\} % right brace in a string
\def\RQ{{\tt\char'23}} % right quote in a string
\def\SP{{\tt\char`\ }} % (visible) space in a string
\chardef\TL=`\~ % tilde in a string
\chardef\UL=`\_ % underline character in a string

\newbox\bak \setbox\bak=\hbox to -1em{} % backspace one em
\newbox\bakk\setbox\bakk=\hbox to -2em{} % backspace two ems

\newcount\ind % current indentation in ems
\def\1{\global\advance\ind by1\hangindent\ind em} % indent one more notch
\def\2{\global\advance\ind by-1} % indent one less notch
\def\3#1{\hfil\penalty#10\hfilneg} % optional break within a statement
\def\4{\copy\bak} % backspace one notch
\def\5{\hfil\penalty-1\hfilneg\kern2.5em\copy\bakk\ignorespaces}% optional break
\def\6{\ifmmode\else\par % forced break
  \hangindent\ind em\noindent\kern\ind em\copy\bakk\ignorespaces\fi}
\def\7{\Y\6} % forced break and a little extra space

\let\yskip=\smallskip
\def\to{\mathrel{.\,.}} % double dot, used only in math mode
\def\note#1#2.{\Y\noindent{\hangindent2em%
    \baselineskip10pt\eightrm#1~\ifacro{\pdfnote#2.}\else#2\fi.\par}}

\newtoks\toksA \newtoks\toksB \newtoks\toksC \newtoks\toksD
\newtoks\toksE \newtoks\toksF \newtoks\usersanitizer
\newcount\countA \countA=0 \newcount\countB \countB=0
\newcount\countC \countC=0
\newif\iftokprocessed \newif\ifTnum \newif\ifinstr
{\def\\{\global\let\spacechar= }\\ }

\ifacro % The following are pdf macros
\def\thewidth{\the\wd0 \space}
\def\theheight{\the\ht\strutbox\space}
\def\thedepth{\the\dp\strutbox\space}
\ifpdftex
  \ifx\pdfannotlink\undefined\let\pdfannotlink\pdfstartlink\fi% for pdfTeX 0.14
  \def\pdflink#1#2{\hbox{\pdfannotlink height\ht\strutbox depth\dp\strutbox
    attr{/Border [0 0 0]} goto num #1 \Blue #1\Black\pdfendlink}}
\else\def\pdflink#1#2{\setbox0=\hbox{\special{pdf: bc [ \pdflinkcolor ]}{#1}%
    \special{pdf: ec}}\special{pdf: ann width \thewidth height \theheight
      depth \thedepth << /Type /Annot /Subtype /Link
      /Border [0 0 0] /A << /S /GoTo /D (#2) >> >>}\box0\relax}\fi
\def\pdfnote#1.{\setbox0=\hbox{\toksA={#1.}\toksB={}\maketoks}\the\toksA}
\def\firstsecno#1.{\setbox0=\hbox{\toksA={#1.}\toksB={}%
    \def\makenote{\addtokens\toksB{\the\toksC}\def\makenote{\toksD={}
      \toksC={}\let\space\empty}\makenote}\maketoks}}
\def\addtokens#1#2{\edef\addtoks{\noexpand#1={\the#1#2}}\addtoks}
\def\poptoks#1#2|ENDTOKS|{\let\first=#1\toksD={#1}%
  \ifcat\noexpand\first0\countB=`#1\else\countB=0\fi\toksA={#2}}
\def\maketoks{\expandafter\poptoks\the\toksA|ENDTOKS|%
  \ifnum\countB>`9 \countB=0 \fi
  \ifnum\countB<`0
    \ifnum0=\countC\else\makenote\fi
    \ifx\first.\let\next=\maketoksdone\else
        \let\next=\maketoks
        \addtokens\toksB{\the\toksD}
        \ifx\first,\addtokens\toksB{\space}\fi
    \fi
  \else \addtokens\toksC{\the\toksD}\global\countC=1\let\next=\maketoks
  \fi
  \next
}
\def\makenote{\addtokens{\toksB}%
    {\noexpand\pdflink{\the\toksC}{\romannumeral\the\toksC}}\toksC={}\global\countC=0}
\def\maketoksdone{\edef\st{\global\noexpand\toksA={\the\toksB}}\st}
\def\pdfURL#1#2{\ifpdftex\pdfannotlink height\ht\strutbox depth\dp\strutbox
  attr {/Border [0 0 0]} user { /Type /Action /Subtype /Link /A
      << /S /URI /URI (#2) >>}\Blue #1\Black \pdfendlink
  \else \ifpdf{\setbox0=\hbox{\special{pdf: bc [ \pdflinkcolor ]}{#1}%
    \special{pdf: ec}}\special{pdf: ann width \thewidth\space height \theheight
      \space depth \thedepth\space << /Border [0 0 0]
      /Type /Action /Subtype /Link /A << /S /URI /URI (#2) >> >>}\box0\relax}%
  \else #1 ({\tt#2})\fi\fi}
{\catcode`\~=12 \gdef\TILDE/{~}} % ~ in a URL
{\catcode`\_=12 \gdef\UNDER/{_}} % _ in a URL
\def\sanitizecommand#1#2{\addtokens\usersanitizer
       {\noexpand\dosanitizecommand\noexpand#1{#2}}}
\def\dosanitizecommand#1#2{\ifx\nxt#1\addF{#2}\fi}

\catcode`\[=1 \catcode`\]=2 \catcode`\{=12 \catcode`\}=12
  \def\lbchar[{]  \def\rbchar[}]
\catcode`\[=12 \catcode`\]=12 \catcode`\{=1 \catcode`\}=2
\catcode`\~=12 \def\tildechar{~}  \catcode`\~=13
\catcode`\|=0 |catcode`|\=12 |def|bschar{\} |catcode`|\=0 \catcode`\|=12
\def\makeoutlinetoks{\Tnumfalse\afterassignment\makeolproctok\let\nxt= }
\def\makeolnexttok{\afterassignment\makeolproctok\let\nxt= }
\def\makeolgobbletok{\afterassignment\makeolnexttok\let\nxt= }
\def\addF#1{\addtokens\toksF{#1}\tokprocessedtrue}
% now comes a routine to "sanitize" section names, for pdf outlines
\def\makeolproctok{\tokprocessedfalse
  \let\next\makeolnexttok % default
  \ifx\nxt\outlinedone\let\next\outlinedone
  \else\ifx{\nxt \else\ifx}\nxt \Tnumfalse \instrfalse % skip braces
  \else\ifx$\nxt % or a $ sign
  \else\ifx^\nxt \addF^\else\ifx_\nxt \addF_% sanitize ^ and _
  \else\ifx\nxt\spacechar \addF\space
  \else\if\noexpand\nxt\relax % we have a control sequence; is it one we know?
    \ifx\nxt~\addF\space
    \else\ifx\nxt\onespace\addF\space
    \else\the\usersanitizer
    \iftokprocessed\else\makeolproctokctli
    \iftokprocessed\else\makeolproctokctlii
    \iftokprocessed\else\makeolproctokctliii % if not recognised, skip it
    \fi\fi\fi\fi\fi
   \else  % we don't have a control sequence, it's an ordinary char
    \ifx/\nxt \addF{\string\/}% quote chars special to PDF with backslash
    \else\ifx(\nxt \addF{\string\(}\else\ifx)\nxt \addF{\string\)}%
    \else\ifx[\nxt \addF{\string\[}\else\ifx]\nxt \addF{\string\]}%
    \else\expandafter\makeolproctokchar\meaning\nxt
   \fi\fi\fi\fi\fi\fi\fi\fi\fi\fi\fi\fi\fi
  \next
}
\def\makeolproctokchar#1 #2 #3{\addF{#3}}
\def\makeolproctokctli{%
  \ifx\nxt\CEE\addF{C}\let\next\makeolgobbletok % \CEE/
  \else\ifx\nxt\UNIX\addF{UNIX}\let\next\makeolgobbletok % \UNIX/
  \else\ifx\nxt\TEX\addF{TeX}\let\next\makeolgobbletok % \TEX/
  \else\ifx\nxt\TeX\addF{TeX}\else\ifx\nxt\LaTeX\addF{LaTeX}%
  \else\ifx\nxt\CPLUSPLUS\addF{C++}\let\next\makeolgobbletok % \CPLUSPLUS/
  \else\ifx\nxt\Cee\addF{C}%
  \else\ifx\nxt\PB \let\next\makeolgobbletok \tokprocessedtrue % \PB{...}
  \else\ifx\nxt\.\tokprocessedtrue\instrtrue % \.{...}
      % skip \|
  \else\ifx\nxt\\\ifinstr\addF{\bschar\bschar}\else\tokprocessedtrue\fi
  \else\ifx\nxt\&\ifinstr\addF&\else\tokprocessedtrue\fi
  \else\ifx\nxt\~\ifTnum\addF{0}\else\addF\tildechar\fi % 077->\T{\~77}
  \else\ifx\nxt\_\ifTnum\addF{E}\else\addF_\fi  % 0.1E5->\T{0.1\_5}
  \else\ifx\nxt\^\ifTnum\addF{0x}\else\addF^\fi  % 0x77 -> \T{\^77}
  \else\ifx\nxt\$\ifTnum\tokprocessedtrue\else\addF$\fi % \T{77\$L}
  \else\ifx\nxt\{\addF\lbchar       \else\ifx\nxt\}\addF\rbchar
  \else\ifx\nxt\ \addF\space        \else\ifx\nxt\#\addF{\string\#}%
  \else\ifx\nxt\PP\addF{++}\else\ifx\nxt\MM\addF{--}%
  \fi\fi\fi\fi\fi\fi\fi\fi\fi\fi\fi\fi\fi\fi\fi\fi\fi\fi\fi\fi\fi
}
\def\makeolproctokctlii{%
  \ifx\nxt\MG\addF{->}\else\ifx\nxt\GG\addF{>>}%
  \else\ifx\nxt\LL\addF{<<}\else\ifx\nxt\NULL\addF{NULL}%
  \else\ifx\nxt\AND\addF&\else\ifx\nxt\OR\addF|%
  \else\ifx\nxt\XOR\addF^\else\ifx\nxt\CM\addF\tildechar
  \else\ifx\nxt\MOD\addF{\string\%}\else\ifx\nxt\DC\addF{::}%
  \else\ifx\nxt\PA\addF{.*}\else\ifx\nxt\MGA\addF{->*}%
  \else\ifx\nxt\this\addF{this}\else\ifx\nxt\?\addF?%
  \else\ifx\nxt\E\addF{==}\else\ifx\nxt\G\addF{>=}%
  \else\ifx\nxt\I\addF{!=}\else\ifx\nxt\K\addF{=}%
  \else\ifx\nxt\l\addF{l}\else\ifx\nxt\L\addF{L}%
  \else\ifx\nxt\o\addF{o}\else\ifx\nxt\O\addF{O}%
  \else\ifx\nxt\R\addF!%
  \else\ifx\nxt\T \Tnumtrue \let\next\makeolgobbletok
    \tokprocessedtrue % \T{number}
  \else\ifx\nxt\AM\addF&\else\ifx\nxt\%\addF{\string\%}%
  \fi\fi\fi\fi\fi \fi\fi\fi\fi\fi
  \fi\fi\fi\fi\fi \fi\fi\fi\fi\fi
  \fi\fi\fi\fi\fi\fi
}
\def\makeolproctokctliii{%
  \ifx\nxt\V\addF{||}\else\ifx\nxt\W\addF{&&}\else\ifx\nxt\Z\addF{<=}%
  \else\ifx\nxt\*\addF*\else\ifx\nxt\Xand\addF{\space and\space}%
  \else\ifx\nxt\Xandxeq\addF{\space and_eq\space}%
  \else\ifx\nxt\Xbitand\addF{\space bitand\space}%
  \else\ifx\nxt\Xbitor\addF{\space bitor\space}%
  \else\ifx\nxt\Xcompl\addF{\space compl\space}%
  \else\ifx\nxt\Xnot\addF{\space not\space}%
  \else\ifx\nxt\Xnotxeq\addF{\space not_eq\space}%
  \else\ifx\nxt\Xor\addF{\space or\space}%
  \else\ifx\nxt\Xorxeq\addF{\space or_eq\space}%
  \else\ifx\nxt\Xxor\addF{\space xor\space}%
  \else\ifx\nxt\Xxorxeq\addF{\space xor_eq\space}%
  \fi\fi\fi\fi\fi\fi\fi\fi\fi\fi
  \fi\fi\fi\fi\fi
}
\def\outlinedone{\edef\outlinest{\global\noexpand\toksE={\the\toksF}}%
  \outlinest\let\outlinedone=\relax}
\fi % End of pdf macros

\def\lapstar{\rlap{*}}
\def\stsec{\Q\noindent{\let\*=\lapstar\bf\modstar.\quad}%
  \ifpdftex\smash{\raise\baselineskip\hbox to0pt{%
     \let\*=\empty\pdfdest num \modstar fith}}
  \else\ifpdf\smash{\raise\baselineskip\hbox to0pt{%
     \let\*=\empty\special{%
       pdf: dest (\romannumeral\modstar) [ @thispage /FitH @ypos ]}}}\fi\fi}
\let\startsection=\stsec
\def\defin#1{\global\advance\ind by 2 \1\&{#1 }} % begin `define' or `format'
\def\A{\note{See also section}} % crossref for doubly defined section name
\def\As{\note{See also sections}} % crossref for multiply defined section name
\def\B{\mathopen{\.{@\{}}} % begin controlled comment
\def\C#1{\ifmmode\gdef\XX{\null$\null}\else\gdef\XX{}\fi %$% Pascal comments
  \XX\hfil\penalty-1\hfilneg\quad$\{\,$#1$\,\}$\XX}
\def\D{\defin{define}} % macro definition
\def\E{\cdot10^} % exponent in floating point constant
\def\ET{ and~} % conjunction between two section numbers
\def\ETs{, and~} % conjunction between the last two of several section numbers
\def\F{\defin{format}} % format definition
\let\G=\ge % greater than or equal sign
\def\H#1{\hbox{\rm\char"7D\tt#1}} % hexadecimal constant
\let\I=\ne % unequal sign
\def\J{\.{@\&}} % TANGLE's join operation
\let\K=\gets % left arrow
\let\L=\le % less than or equal sign
\outer\def\M#1.{\MN#1.\ifon\vfil\penalty-100\vfilneg % beginning of section
  \vskip12ptminus3pt\startsection\ignorespaces}
\outer\def\N#1.#2.{\MN#1.\vfil\eject % beginning of starred section
  \def\rhead{\uppercase{\ignorespaces#2}} % define running headline
  \message{*\modno} % progress report
  \edef\next{\write\cont{\Z{#2}{\modno}{\the\pageno}}}\next % to contents file
  \ifon\startsection{\bf\ignorespaces#2.\quad}\ignorespaces}
\def\MN#1.{\par % common code for \M, \N
  {\xdef\modstar{#1}\let\*=\empty\xdef\modno{#1}}
  \ifx\modno\modstar \onmaybe \else\ontrue \fi \mark{\modno}}
\def\O#1{\hbox{\rm\char'23\kern-.2em\it#1\/\kern.05em}} % octal constant
\def\P{\rightskip=0pt plus 100pt minus 10pt % go into Pascal mode
  \sfcode`;=3000
  \pretolerance 10000
  \hyphenpenalty 10000 \exhyphenpenalty 10000
  \global\ind=2 \1\ \unskip}
\def\Q{\rightskip=0pt % get out of Pascal mode
  \sfcode`;=1500 \pretolerance 200 \hyphenpenalty 50 \exhyphenpenalty 50 }
\let\R=\lnot % logical not
\let\S=\equiv % equivalence sign
\def\T{\mathclose{\.{@\}}}} % terminate controlled comment
\def\U{\note{This code is used in section}} % crossref for use of a section
\def\Us{\note{This code is used in sections}} % crossref for uses of a section
\let\V=\lor % logical or
\let\W=\land % logical and
\def\X#1:#2\X{\ifmmode\gdef\XX{\null$\null}\else\gdef\XX{}\fi %$% section name
  \XX$\langle\,$#2{\eightrm\kern.5em
    \ifacro{\pdfnote#1.}\else#1\fi}$\,\rangle$\XX}
\def\Y{\par\yskip}
\let\Z=\let % now you can \send the control sequence \Z
\def\){\hbox{\.{@\$}}} % sign for string pool check sum
\def\]{\hbox{\.{@\\}}} % sign for forced line break
\def\=#1{\kern2pt\hbox{\vrule\vtop{\vbox{\hrule
        \hbox{\strut\kern2pt\.{#1}\kern2pt}}
      \hrule}\vrule}\kern2pt} % verbatim string
\let\~=\ignorespaces
\let\*=*

\def\onmaybe{\let\ifon=\maybe} \let\maybe=\iftrue
\newif\ifon \newif\iftitle \newif\ifpagesaved
\def\lheader{\mainfont\the\pageno\eightrm\qquad\rhead\hfill\title\qquad
  \tensy x\mainfont\topmark} % top line on left-hand pages
\def\rheader{\tensy x\mainfont\topmark\eightrm\qquad\title\hfill\rhead
  \qquad\mainfont\the\pageno} % top line on right-hand pages
\def\page{\box255 }
\def\normaloutput#1#2#3{\ifodd\pageno\hoffset=\pageshift\fi
  \shipout\vbox{
    \vbox to\fullpageheight{
      \iftitle\global\titlefalse
      \else\hbox to\pagewidth{\vbox to10pt{}\ifodd\pageno #3\else#2\fi}\fi
      \vfill#1}} % parameter #1 is the page itself
  \global\advance\pageno by1}

\def\rhead{\.{WEB} OUTPUT} % this running head is reset by starred sections
\def\title{} % an optional title can be set by the user
\def\topofcontents{\centerline{\titlefont\title}
  \vfill} % this material will start the table of contents page
\def\botofcontents{\vfill} % this material will end the table of contents page
\def\contentspagenumber{0} % default page number for table of contents
\newdimen\pagewidth \pagewidth=6.5in % the width of each page
\newdimen\pageheight \pageheight=8.7in % the height of each page
\newdimen\fullpageheight \fullpageheight=9in % page height including headlines
\newdimen\pageshift \pageshift=0in % shift righthand pages wrt lefthand ones
\def\magnify#1{\mag=#1\pagewidth=6.5truein\pageheight=8.7truein
  \fullpageheight=9truein\setpage}
\def\setpage{\hsize\pagewidth\vsize\pageheight} % use after changing page size
\def\contentsfile{CONTENTS} % file that gets table of contents info
\def\readcontents{\input CONTENTS}

\newwrite\cont
\output{\setbox0=\page % the first page is garbage
  \openout\cont=\contentsfile
  \global\output{\normaloutput\page\lheader\rheader}}
\setpage
\vbox to \vsize{} % the first \topmark won't be null

\def\ch{\note{The following sections were changed by the change file:}
  \let\*=\relax}
\newbox\sbox % saved box preceding the index
\newbox\lbox % lefthand column in the index
\def\inx{\par\vskip6pt plus 1fil % we are beginning the index
  \write\cont{} % ensure that the contents file isn't empty
  \closeout\cont % the contents information has been fully gathered
  \output{\ifpagesaved\normaloutput{\box\sbox}\lheader\rheader\fi
    \global\setbox\sbox=\page \global\pagesavedtrue}
  \pagesavedfalse \eject % eject the page-so-far and predecessors
  \setbox\sbox\vbox{\unvbox\sbox} % take it out of its box
  \vsize=\pageheight \advance\vsize by -\ht\sbox % the remaining height
  \hsize=.5\pagewidth \advance\hsize by -10pt
    % column width for the index (20pt between cols)
  \parfillskip 0pt plus .6\hsize % try to avoid almost empty lines
  \def\lr{L} % this tells whether the left or right column is next
  \output{\if L\lr\global\setbox\lbox=\page \gdef\lr{R}
    \else\normaloutput{\vbox to\pageheight{\box\sbox\vss
        \hbox to\pagewidth{\box\lbox\hfil\page}}}\lheader\rheader
    \global\vsize\pageheight\gdef\lr{L}\global\pagesavedfalse\fi}
  \message{Index:}
  \parskip 0pt plus .5pt
  \outer\def\:##1, ##2.{\par\hangindent2em\noindent##1:\kern1em
    \ifacro\pdfnote##2.\else##2\fi.} % index entry
  \let\ttentry=\. \def\.##1{\ttentry{##1\kern.2em}} % give \tt a little room
  \def\[##1]{$\underline{##1}$} % underlined index item
  \rm \rightskip0pt plus 2.5em \tolerance 10000 \let\*=\lapstar
  \hyphenpenalty 10000 \parindent0pt}
\def\fin{\par\vfill\eject % this is done when we are ending the index
  \ifpagesaved\null\vfill\eject\fi % output a null index column
  \if L\lr\else\null\vfill\eject\fi % finish the current page
  \parfillskip 0pt plus 1fil
  \def\rhead{NAMES OF THE SECTIONS}
  \message{Section names:}
  \output{\normaloutput\page\lheader\rheader}
  \setpage
  \def\note##1##2.{\hfil\penalty-1\hfilneg\quad{\eightrm##1~%
     \ifacro{\pdfnote##2.}\else{##2}\fi.}}
  \linepenalty=10 % try to conserve lines
  \def\U{\note{Used in section}} % crossref for use of a section
  \def\Us{\note{Used in sections}} % crossref for uses of a section
  \def\:{\par\hangindent 2em}\let\*=*\let\.=\ttentry}
\def\con{\par\vfill\eject % finish the section names
  \rightskip 0pt \hyphenpenalty 50 \tolerance 200
  \setpage
  \output{\normaloutput\page\lheader\rheader}
  \titletrue % prepare to output the table of contents
  \pageno=\contentspagenumber \def\rhead{TABLE OF CONTENTS}
  \message{Table of contents:}
  \topofcontents
  \line{\hfil Section\hbox to3em{\hss Page}}
  \def\Z##1##2##3{\line{\ignorespaces##1
      \rm\leaders\hbox to .5em{.\hfil}\hfil
      \ \ifacro\pdflink{##2}{\romannumeral##2}\else##2\fi
      \hbox to3em{\hss##3}}}
  \readcontents\relax % read the contents info
  \botofcontents\makeoutlines\end} % print the contents page(s) and terminate
\ifpdftex\def\makeoutlines{%
  \def\?##1]{}\def\Z##1##2##3{\pdfoutline goto num ##2{##1}}
  \input CONTENTS\relax}
  \pdfcatalog{/PageMode /UseOutlines}
\else\let\makeoutlines=\relax \fi
\endinput
